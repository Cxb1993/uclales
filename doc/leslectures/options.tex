\author{Thijs Heus}
\lecture{Model Options}{options}
\begin{frame}{Starting a model run}
There are four ingredients that feed into the model
\begin{itemize}
 \item Hardcoded options
 \item Restart files (in NetCDF format)
 \item Data files (in text format)
 \item An option file: NAMELIST
\end{itemize}
\end{frame}

\begin{frame}[allowframebreaks]{Available runs}
 In \code{misc/initfiles} the following cases are provided by default:

\begin{itemize}
\item \textbf{namelist\_astex}: The Astex case.
\item \textbf{namelist\_cumulus}: Namelist to reproduce the idealized
cumulus cases reported in Stevens, JAS (2007). Requires the
generation of a sound\_in file with bstate.f95.
\item \textbf{namelist\_drycbl}: Idealized dry CBL consisting of a
layer with initially uniform stratification and constant forcing.
\item \textbf{namelist\_dycm01}: The DYCOMS GCSS RF01 case, requires
the generation of a sound\_in file with bstate.f95.
\item \textbf{namelist\_dycm02}: The DYCOMS GCSS RF02 case, requires
the generation of a sound\_in file with bstate.f95, as well as the
generation of zm\_grid\_in and zt\_grid\_in files uzing zgrid.gcss9.f.
\item \textbf{namelist\_rico}: The RICO GCSS composite case.
\item \textbf{namelist\_smoke}: The GCSS smoke case.
\end{itemize}
\end{frame}

\section{Data files}
\begin{frame}{Data files}
\framesubtitle{*\_grid\_in}
\begin{itemize}
 \item \code{zm\_grid\_in}, \code{zt\_grid\_in} Input files for vertical non-equidistant grids that are not possible with the namelist options. A single column of values, needs to have at least nzp-2 points
\end{itemize}

\end{frame}

\begin{frame}{Data files}
\framesubtitle{sound\_in}
\begin{itemize}
 \item A completely flexible input file for the initial profiles of the mean quantities
 \item Textfile with a bunch of rows:
\begin{itemize}
 \item height in meters or in pressure (depending on \code{ipsflg}) The first number is the surface pressure
 \item Temperature. Depending on \code{itsflg}, the absolute temperature, potential temperature or liquid water potential temperature. 
 \item Humidity. Depending on \code{irsflg}, the relative humidity or total humidity
 \item Horizontal velocity fields, $u$ and $v$.
\end{itemize}
The file contents should cover the entire domain. Between anchor points, linear interpolation happens.
\end{itemize}
\end{frame}

\begin{frame}{Data files}
\framesubtitle{ls\_flux\_in} 
Time dependent fluxes and large scale forcings. 
\begin{itemize}
 \item The first block sets the surface values, with columns:
\begin{itemize}
 \item Time in seconds
 \item 	Surface heat flux in $W m^{-2}$
 \item 	Surface moisture flux in $W m^{-2}$
 \item Surface liquid water potential temperature
 \item Surface pressure
\end{itemize}
 \item From the  second block on, every block starts with: \code{\# time}
 \item Within each block, the following columns show up:
 \begin{itemize}
   \item Large scale subsidence $w_s$, gives the tendency $-w_s\derr{\phi}{z}$
   \item Large scale tendency for $\theta_l$
   \item Large scale tendency for $q_t$
  \end{itemize}
\end{itemize}
The block contents should cover the entire domain. Between anchor points, linear interpolation happens.
\end{frame}
 
\begin{frame}{Data files}
\framesubtitle{nudge\_in}
Nudges the average fields to a preset value:
\[
 \left.\derr{\phi}{t}\right| = \frac{\phi_{nudge} - \bar{\phi}}{\tau}
\]
 With $\tau^{-1}$ the nudging strength.


The columns depict:
\begin{itemize}
 \item height in meters
 \item Nudging strength
 \item The nudging value of $u$, $v$, $\theta_l$ and $q_t$
\end{itemize}

The nudging can be time dependent, so each block shows the nudging at a specific time, set by the number that starts the block just after the \#
\end{frame}

\begin{frame}{Data files}
\framesubtitle{datafiles directory}
\begin{itemize}
 \item \code{dmin\_wetgrowth\_lookup.dat} Only for level=5 microphysics: Look up table for growth ice hydrometeors
 \item \code{*.lay}: To be copied to the run dirs and named \code{backrad\_in}. It describes the radiative background state of the atmosphere, including pressure, temperature, humidity and ozone profiles. Only used for \code{iradtyp = 4} and between the top of the domain and the top of the atmosphere. 
 \item \code{*.dat} Internal lookup tables for iradtyp=4 radiation
\end{itemize}
\end{frame}

\section[Namelist]{NAMELIST options}
\begin{frame}{The Namelist}
\begin{itemize}
 \item The only obligatory input file
 \item Has to be named: \code{NAMELIST} (in capitals)
 \item All input is being put in a single namelist, read at \code{LES.f90}
\end{itemize}
\end{frame}
\begin{frame}[allowframebreaks]{Grid and Time setup}
\mylineno=0\begin{longtable}{p{0.1\linewidth}p{0.15\linewidth}p{0.65\linewidth}}
\alert{Variable} &\alert{Default} & \endhead
expnme  &  Default & experiment name \tblnewline
filprf  &  x & file prefix for use in constructing output files\tblnewline
nxp & 132 & total number of $x$ points ($N_y+4$) \tblnewline
nyp &   132 & total number of $y$ points ($N_y+4$) \tblnewline
nzp &   105 & total number of $z$ points \tblnewline
deltax  & 35.0 m & grid spacing in $x$-direction \tblnewline
deltay  & 35.0 m & grid spacing in $y$-direction \tblnewline
deltaz  & 17.5 m & grid spacing in $z$-direction \tblnewline
dzrat   & 1.02   & grid stretching ration (default 2\% per
interval) \tblnewline
dzmax  &  1200 m & height at which grid-stretching begins \tblnewline
igrdtyp &  1 & control parameter for selecting vertical grid \tblnewline
dtlong &  10 s  & maximum timestep \tblnewline
hfilin &  test. & name of input history file for HISTORY starts (xxx.)\tblnewline
timmax & 18000 s  &  final time of simulation \tblnewline 
wctime & & Wall clock time to break off the simulation \tblnewline
nfpt   &  5      & number of levels in upper sponge layer \tblnewline
distim &  300 s  & minimum relaxation time in sponge layer \tblnewline
% \tblnewline
naddsc  &  0 &  number of additional scalars \tblnewline
runtyp  &  INITIAL & type of run ('INITIAL' or 'HISTORY') \tblnewline 
\end{longtable}
\end{frame}
\begin{frame}[allowframebreaks]{Physics}
\mylineno=0\begin{longtable}{p{0.15\linewidth}p{0.15\linewidth}p{0.6\linewidth}}
\alert{Variable} &\alert{Default} & \tblnewline 
\endhead
iradtyp &  0 & control parameter for selecting radiation model \tblnewline
CCN     &  150 $\times 10^{6}$ & cloud droplet mixing ratio \tblnewline
level   &  0 & 0=thermodynamic level, 1=dry cbl, 2=moist cbl (no rain), 3=moist cbl (with rain), 4, 5= ice microphysics\tblnewline
corflg    &  false   &  coriolis acceleration (true/false) \tblnewline
radfrq    &  0   &  radiation update interval \tblnewline
strtim    &  0   &  GMT of model time \tblnewline
cntlat    &  31.5$^{\circ}$ N &  model central latitude \tblnewline
case\_name    &  astex &  specify case name (rico,astex,bomex)\tblnewline
lsvarflg  &  false & reads large scale forcings from the file lscale\_in\tblnewline
div       &  3.75e-6 s$^{-1}$ & divergence\tblnewline
umean     & 0. & Mean $U$ velocity (subtracted during the calculations \tblnewline
vmean     & 0. & Mean $V$ velocity (subtracted during the calculations \tblnewline
th00      & 288 & Basic state temperature (subtracted during the calculations \tblnewline
sst       &  292 K &  sea surface temperature \tblnewline 
isfctyp   & 0  &  surface parameterization type (0: specified
fluxes; 1: specified surface layer gradients; 2: fixed lower boundary
of water, 3-5: Specific variations. See the surface lecture for more information.\tblnewline
 ubmin     &  0.20  &  minimum $u$ for $u_*$ computation \tblnewline
 zrough    &  0.1 &  momentum roughness height (if less than
 zero use Charnock relation)\tblnewline
 dthcon    & 100 Wm$^{-2}$ &  surface temperature gradient
 (isfcflg=1) or surface heat flux (itsflg=0) \tblnewline
 drtcon    & 0   Wm$^{-2}$ &   surface humidity (mixing ratio) gradient
 (isfcflg=1) or surface latent heat flux (itsflg=0) \tblnewline
 csx       &   0.23   &  Smagorinsky Coefficient \tblnewline
 prndtl    &    1/3    &  Prandtl Number (if less than zero no sgs for scalars) \tblnewline
sfc\_albedo && Albedo of the surface \tblnewline
lnudge && Switching on/off nudging \tblnewline
 tnudgefac && Factor to strengthen the nudging \tblnewline
 ltimedep && Switch for time depend fluxes and large scale forcings \tblnewline
   SolarConstant && Top of Atmosphere radiation \tblnewline

\end{longtable}
\end{frame}
\begin{frame}[allowframebreaks]{Initial profiles}
\mylineno=0\begin{longtable}{p{0.15\linewidth}p{0.15\linewidth}p{0.6\linewidth}}
\alert{Variable} &\alert{Default} & \tblnewline 
\endhead
ipsflg&  1   & control parameter for input sounding (0: pressure
in hPa; 1: height in meters with ps(1)= $p_{sfc}$) \tblnewline
itsflg&   1   & control parameter for input sounding (0: ts =
$\theta;$ 1: $ts = \theta_l$) \tblnewline
irsflg&   1   & control parameter for input sounding (0: rs =
Rel. Hum) 1: ($rs = q_t$) \tblnewline
us    &   n/a & input zonal wind souding \tblnewline
vs    &   n/a & input meridional wind souding  \tblnewline
ts    &   n/a & input temperature souding  \tblnewline
rts   &   n/a & input humidity souding  \tblnewline
ps    &   n/a & input pressure sounding \tblnewline
hs    &   n/a & vertical position  \tblnewline
iseed &   0   &  random seed \tblnewline
zrand &   200 m & height below which random perturbations are added \tblnewline
\end{longtable}
\end{frame}
\begin{frame}[allowframebreaks]{Statistics and output}
\mylineno=0\begin{longtable}{p{0.15\linewidth}p{0.15\linewidth}p{0.6\linewidth}}
\alert{Variable} &\alert{Default} & \tblnewline 
\endhead
outflg    &  true   &  output flag (true/false) \tblnewline
lsync & false &  Synchronize the crosssection output (true/false)\tblnewline
frqhis    &  9000 s &  history write interval \tblnewline
frqanl    &  3600 s &  analysis write interval  \tblnewline
slcflg    &  false  &  write slice output (true/false) \tblnewline 
istpfl    &  1 &  print interval for timestep info \tblnewline
ssam\_intvl&    30 s  &  statistics sampling interval\tblnewline
savg\_intvl&  1800 s    &  statistics averaging interval \tblnewline
lcross & false &  Crosssection output (true/false)\tblnewline
frqcross    &  3600 s &  crosssection write interval  \tblnewline
lxy & false &  Crosssection output in xy plane (true/false)\tblnewline
zcross & 0 &  Crosssection location of xy plane (true/false)\tblnewline
lxz & false &  Crosssection output in xz plane (true/false)\tblnewline
ycross & 0 &  Crosssection location of xy plane (true/false)\tblnewline
lyz & false &  Crosssection output in yz plane (true/false)\tblnewline
xcross & 0 &  Crosssection location of xy plane (true/false)\tblnewline
lwaterbudget & false &  Crosssection of (costly) waterbudget (true/false)\tblnewline
\end{longtable}
\end{frame}
