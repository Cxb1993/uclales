
\author{Thijs Heus}
\lecture[Setting up]{Setting up the code: Obtaining, compiling, running (and version management)}{setup}
% \begin{frame}[<+->]{Course Setup}
%  \begin{itemize}
%   \item login on tornado
%   \item \code{cd course/yourname}
%   \item Contents: 
% \begin{itemize}
% \item A public SSH key (for \code{gitorious.org})  
% \item A directory with the lectures (will be updated)
% \item A directory with supplementary material (e.g. articles to read)
% \item A directory to run your runs
%  \end{itemize}
%  \item Do not overwrite these files - they will be updated
%  \item Not yet here: The source code
%  \item \alert{Feel free to do the course on your own account/machine!}
% \end{itemize}
% 
% \end{frame}

\section[Git]{Git version management}
\begin{frame}{Git version management}
 \begin{itemize}
  \item Git is a distributed version management system
  \item All history of all branches is captured
  \item Easy to create branches for some project (like the course)
  \item Easy to merge fixes and features from branch to branch
  \item The main repository sits on \code{www.gitorious.org/uclales}
  \item The \code{master} branch should always be the most stable, up-to-date branch
 \end{itemize}

\end{frame}

\begin{frame}[<+->]{Gitorious.org}
 \begin{itemize}
  \item Register on \code{www.gitorious.org} (already done?)
  \item Tell me your username there, to give you (write) access to UCLALES
  \item Login at \code{www.gitorious.org}
  \item Go to ``Manage SSH keys''
  \item Go to ``Add SSH key''
  \item Add the contents of \code{key/id\_rsa.pub} (or \code{$\sim$/.ssh/id\_rsa.pub}) and click OK
  \item Take some time to browse through the website
 \end{itemize}
\end{frame}
\begin{frame}[<+->]{Using Git}
\framesubtitle{Obtaining the code}
\begin{itemize}
 \item In your course directory, download the code with \code{git clone git@gitorious.org:uclales/uclales.git}
 \item \code{cd uclales; ls}
 \item The entire history is now local in your folder
 \item \code{git branch -a} shows all branches
 \item By default, you are on the master branch
\end{itemize}
\end{frame}
\begin{frame}[<+->]{Using Git}
\framesubtitle{Switching branches}
\begin{itemize}
 \item The start off point for your code is the master branch, so go there if you're not already on it: \code{git checkout master}
%  \item Some differences appear there
 \item Now make your personal branch, based on the course branch: \code{git checkout -b yourname}
 \item Here you can play whatever you like
\end{itemize}
\end{frame}
\begin{frame}[<+->]{Using Git}
\framesubtitle{Changing something}
\begin{itemize}
\item Open the file \code{test1}
\item Write something in it
\item See what is different: \code{git status} and \code{git diff}
\item If you are happy with you change, commit: \code{git commit test1} or \code{git commit -a} for all changes
\item Write a commit message and save
\item See what is different now: \code{git diff}
\item Nothing!
\end{itemize}
\end{frame}
\begin{frame}[<+->]{Using Git}
\framesubtitle{Creating a new file}
\begin{itemize}
\item Open the new file \code{test2}
\item Write something in it
\item See what is different: \code{git status} and \code{git diff}
\item You have to add the file with \code{git add test2}
\item If you are happy with you change, commit: \code{git commit test1} or \code{git commit -a} for all changes
\item Write a commit message and save
\item See what is different now: \code{git diff}
\item Nothing!
\end{itemize}
\end{frame}
\begin{frame}[<+->]{Using Git}
\framesubtitle{Updating the remote repository}
\begin{itemize}
\item On \code{gitorious.org}, nothing has changed yet
\item To update: \code{git push origin yourname}
\item Refresh \code{gitorious.org}; many new branches
\item To get them all: \code{git pull}
\item \code{git branch -a} has more branches now
\end{itemize}
\end{frame}
\begin{frame}[<+->]{Using Git}
\framesubtitle{Other commands}
\begin{itemize}
\item \code{git rm filename} and \code{git mv filename} (Re)move files
\item \code{git merge branchname} merges \code{branchname} into the current branch
\item  \code{git checkout -f filename} resets a single file to whatever was committed
\item \code{git reset} is the panic button and reverts everything to the previous state
\item See \code{uclales/doc/git\_uclales.pdf} for longer explanation
\end{itemize}

\end{frame}

\section{Compilation}
\begin{frame}[<+->]{Compilation}
\framesubtitle{Requirements}
UCLALES requires almost no outside libraries.
 \begin{itemize}
  \item NetCDF (v3 or later) for input and output 
  \item MPI (Only if you want to do Parallel runs)
  \item A Fortran 95 compiler (IFort, gfortran, xlf work)
  \item Git for keeping up to date with the source code
  \item CMake (optional) for easier/faster compilation
 \end{itemize}
On thunder, load cmake, Ifort and mpi with:\\
\code{module load cmake \\ module load intel\/13.0.0 \\ module load openmpi\_ib\/1.6.2-static-intel13}
\end{frame}
\begin{frame}[<+->]{Compilation}
\framesubtitle{Cmake and Make}
There are two ways of compiling the code.
\begin{itemize}
 \item CMake does its best to create a Makefile automatically. 
\begin{itemize}
 \item Allows for parallel compilation
 \item Easier to maintain
 \item Not on every system
%  \item On thunder: \code{module load cmake} to your \code{PATH}
\end{itemize}
 \item A bunch of predefined Makefiles are available in the \code{misc/makefiles} directory.
\end{itemize}
\end{frame}

% \begin{frame}[<+->]{Compilation}
% \framesubtitle{Provided Makefiles}
% The Makefiles provided are platform specific, and has to be maintained whenever internal dependencies change.\begin{itemize}
%  \item For a parallel build: \\ \code{cd uclales/src; cp mpi/mpi\_interface .}
%  \item For a sequential build:\\ \code{cd uclales/src; cp seq/seq\_interface .}
%  \item For both: Copy a main makefile to the bin directory: \\ \code{cd uclales/bin; cp ../misc/makefiles/Makefile\_tornado Makefile}
%  \item Execute make: \code{make seq} or \code{make mpi}
%  \item Executables: \code{les.seq} or \code{les.mpi}
% \end{itemize}
% 
% \end{frame}

\begin{frame}[allowframebreaks]{Compilation}
\framesubtitle{CMake}
\begin{itemize}
 \item The \code{CMakeLists.txt} file in the \code{uclales} dir sets all the options, searches for libraries etc.
 \item Overrides can be set on the commandline or in a configuration file
 \item Choose/edit a configuration file in \code{uclales/config}. This sets paths to libraries 
 \item For now, just copy the thunder one to default: \\ \code{cp thunder.cmake default.cmake}
 \item Create a build directory \\ \code{mkdir build; cd build} from the \code{uclales} dir
 \item Run CMake to create the makefile: \code{cmake -D MPI=FALSE ..}
 \item \code{make -j4} to build the binary \code{uclales}
 \item Executing \code{./uclales} gives an error now: Missing \code{NAMELIST}
\end{itemize}
\end{frame}

\begin{frame}[<+->]{Compilation}
\framesubtitle{CMake options}
CMake responds to a number of commandline options, case sensitive, always with -D as a flag
\mylineno=0\begin{longtable}{p{0.3\linewidth}p{0.15\linewidth}p{0.4\linewidth}}
\alert{Variable} &\alert{Values} & \tblnewline 
\endhead
MPI & TRUE, FALSE & Switch between parallel and serial \tblnewline
CMAKE\_BUILD\_TYPE & DEBUG, RELEASE & Switch between debug settings and optimized \tblnewline
PROFILER & GPROF, SCALASCA, MARMOT & Switch on profiler (to assess speed bottleneck) \tblnewline
\end{longtable}
\end{frame}


\section{Executing}
\begin{frame}[<+->]{Executing}
\framesubtitle{}
\begin{itemize}
 \item Copy the executable \code{uclales} to the \code{run} directory
 \item We need a runscript (\code{uclales/misc/jobscripts/runscript\_course\_seq})
 \item We need a \code{NAMELIST} (\code{uclales/misc/initfiles/namelist\_drycbl})
 \item Submit it: \code{qsub runscript\_course\_seq}
 \item Wait...
 \item See what happens with: \code{tail -f output}
\end{itemize}

\end{frame}
