\documentclass[a4paper,10pt]{article}
%\documentclass[a4paper,10pt]{scrartcl}

\usepackage[utf8]{inputenc}

\title{Hands on with UCLA-LES}
\author{HERZ}
\date{Room 301, Fri, March 15 2013}

\pdfinfo{%
  /Title    ()
  /Author   ()
  /Creator  ()
  /Producer ()
  /Subject  ()
  /Keywords ()
}

\begin{document}
\maketitle
Since we'll be working in room 301, everybody needs to bring their own laptop. All it needs to be able to do is to log in on thunder (or any machine you would like to run on).

\section*{Homework}
To get the maximum out of the course, we'd like participants to get as far as possible in setting up the model by themselves. Please register yourself at www.gitorious.org, and tell me your username so that I can add you to our group. For maximum progress, please see how far you get in following the 2011 lecture notes to retrieve the model using the git version management system, and in building the model. Anything beyond that is bonus, and allows us to help participants with specific questions.

\section*{Schedule}
We hope to have a fairly interactive program, so anything in this schedule is up for last minute change.
\begin{itemize}
 \item[9.30] Introduction: What is LES, what is UCLALES.
 \item[10.00] Retrieving the code: Using the git version management system
 \item[10:45] Building and running: Build the code, do a short test run, and explore the available output
 \item[12:30] Lunch
 \item[13:30] Options and Switches: What's in the Namelists and the input files
 \item[14:30] Hands on session: Modify the statistics to your liking (and commit it to your own git branch)
 \item[17:00] End
\end{itemize}

\end{document}
