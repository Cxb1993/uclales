\author{Thijs Heus}
% \lecture[Cumulus]{BOMEX Shallow Cumulus}{cumulus}
\begin{frame}{BOMEX Shallow Cumulus}
\begin{itemize}
 \item Check \code{articles/siebesma2003.pdf} for the initial settings of BOMEX
 \item Build a \code{NAMELIST} based on it. Hint: the \code{RICO} Namelist should be a good starting point
 \item Run the run, postprocess like the Dry CBL run
 \item If successful, commit your NAMELIST to git
 \item Rerun your run with a different name, but with \code{level=3} for microphysics in the NAMELIST
\end{itemize}
\end{frame}

\begin{frame}[allowframebreaks]{Questions}
 \begin{itemize}
  \item Plot the cloud fraction and the cloud cover. What is the difference between the two?
  \item What are cloud base and cloud top? There are several cloud bases/tops in the \code{.ts} file. What is the difference between them? What can we (implicitly) learn about these clouds based upon these numbers?
  \item One classical way of parametrizing (shallow) cumuli in large scale models, is to model the transport through the cloud layer with a mass flux approach. If necessary, read up on it in \code{siebesma1995.pdf}. They found that entrainment and detrainment rates in the large scale models were off by an order of magnitude.
  \item Try and reproduce figures 6 and on from that study using the output of the \code{.ps} file. \code{\_cs1} is the conditional sampling over the cloud. \code{\_cs2} is the conditional sampling for the buoyant part of the cloud.
  \item BOMEX was an intercomparison case of non-precipitating cumulus clouds. Is the non-precipitating really true, or just because of a lack of microphysical models a decade ago?
  \item  If precipitation is present, does it matter?
 \end{itemize}
\end{frame}
